\documentstyle[12pt]{article}
%
 \voffset=-3cm
 \evensidemargin=0cm
 \oddsidemargin=0cm
 \textwidth=15cm
 \textheight=21cm
%
%
% fonts i simbols
%
\font\tenbbb=msbm10 scaled\magstep1
\font\egtbbb=msbm8
\font\sixbbb=msbm6
\newfam\bbbfam
\textfont\bbbfam=\tenbbb
\scriptfont\bbbfam=\egtbbb
\scriptscriptfont\bbbfam=\sixbbb
\newcommand{\Bbb}[1]{{\fam\bbbfam\relax#1}}
\newcommand{\ZZ}{{\Bbb Z}}
\newcommand{\sprod}[2]{\left\langle#1,#2\right\rangle} % scalar product

\begin{document}
\parindent=0pt

This vectorfield models the motion of an infinitessimal particle
moving close to the equilateral points of the Earth-Moon system:
\begin{eqnarray*}
\ddot{x} & = & P(7)\left[-\frac{x-x_E}{r_{PE}^3}(1-\mu_M)-
         \frac{x+x_E}{r_{PM}^3}\mu_M-x_E(1-2\mu_M)\right]+  \\
     & & +P(1)+P(2)x+P(3)y+P(4)z+P(5)\dot{x}+P(6)\dot{y},      \\
\ddot{y} & = & P(7)\left[-\frac{y-y_E}{r_{PE}^3}(1-\mu_M)-
         \frac{y-y_E}{r_{PM}^3}\mu_M-y_E\right]+P(8)+P(9)x+        \\
     & & +P(10)y+P(11)z+P(12)\dot{x}+P(13)\dot{y}+P(14)\dot{z},\\
\ddot{z} & = & P(7)\left[-\frac{z}{r_{PE}^3}(1-\mu_M)-
         \frac{z}{r_{PM}^3}\mu_M\right]+P(15)+P(16)x+P(17)y+      \\
     & & +P(18)z+P(19)\dot{y}+P(20)\dot{z},
\end{eqnarray*}
where $r_{PE}$, $r_{PM}$ denote the distances from the particle to the
Earth and Moon, respectively, given by $r_{PE}^2=(x-x_E)^2+(y-y_E)^2+
z^2$, $r_{PM}^2=(x+x_E)^2+(y-y_E)^2+z^2$. We recall that $x_E=-1/2$,
$y_E=-\sqrt{3}/2$ for $L_4$ and $x_E=-1/2$, $y_E=\sqrt{3}/2$ for
$L_5$. Finally, $\mu_M$ is $0.012150298$.

The symbols $P(i)$ that appear in the equations denote functions that
(only) depend on time in a quasiperiodic way. The are of the form:
$$
P(i) = A_{i,0} + \sum_{j=1}^{m}A_{i,j}\cos\theta_j+
       \sum_{j=1}^{m}B_{i,j}\sin\theta_j,
$$
with $\theta _{j} =\sprod{k^{(j)}}{\phi}$, where $k^{(j)}\in\ZZ^5$ and
$\phi=(\phi_1,\ldots,\phi_5)$, being $\phi_l=\omega_l t+\varphi_l$
($t$ denotes the time).

\bigskip


\end{document}
